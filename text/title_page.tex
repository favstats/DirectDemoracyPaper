
%<!-- Titleseite -->
\thispagestyle{empty}

%%% redefine \maketitle
\renewcommand{\maketitle}{
	\begin{titlepage}
		\begin{center}
			\setlength{\parskip}{0pt}
			
			%	    \begin{flushright}
			%	    \colorbox{darkgray}{\color{white}{\Large \textsf{\@headerimg}}}
			%             \end{flushright}
			\begin{multicols}{2}
				\flushleft
				{Prof. Dr. Angelika Vetter\par}
				%{Seminar: Representative, direct and\\ cooperative participation in comparison\par}
				{Institute for Social Sciences\par}
				{Department of Political Systems \&\par}
				{Political Sociology}
				\begin{flushright}
					\includegraphics[width=7cm]{images/logo_stuttgart.jpg}
				\end{flushright}
			\end{multicols}
			\vspace*{2mm}
			\center
			{\LARGE {Seminar Paper} \par}
			
			\vspace*{10mm}
			
			
			{\fontsize{26}{38} {\bfseries Measurement Approaches of Direct Democracy} \par}
			\vspace*{1mm}
			{ \Large A Cross-National Theoretical and Empirical Examination}
			\vspace*{10mm}
			
			\begin{multicols}{2}
				\center
				Author: Fabio Votta, B.A.\\
				Email: \href{mailto:fabio.votta@gmail.com}{fabio.votta@gmail.com}\\
				Student ID: 2891518\\
				
				Author: Rosa Seitz, B.A.\\
				Email: \href{mailto:rosa.marie.seitz@gmail.com}{rosa.marie.seitz@gmail.com}\\
				Student ID: 2876533\\
			\end{multicols}
			
			
			\vspace*{5mm}
			
			
			
			
			
			\vspace*{5mm}
			{Date of Submission: 03.04.2018 \par} %\date{xxx}
			
		\end{center}
		\vspace*{2mm}
		\begin{abstract}
			\justifying
			\noindent The aim of this work is to theoretically and empirically compare selected approaches for the measurement of direct democracy as well as to examine available data sources on direct democratic institutions in regard to their usability for quantitative research. To this end, a range of criteria have been identified and different databases for direct democratic institutions are selected and compared on the basis of whether they provide information on these criteria. Next, contemporary approaches assessing direct democracy are examined in whether they cover such criteria like decisiveness and easiness of approval/initiation and in which way they are accounted for. During the examination, it is differentiated between approaches assessing rules in form and  rules in use and measures that include a combination of both. Another crucial differentiation which is regarded whenever possible, is the one between bottom-up and top-down mechanisms of direct democracy. In the empirical comparison, we find that different direct democracy measures generally correlate highly, with mostly minor differences. Although it has to be noted that we only compared rules in form, rules in use, and mixed assessments within each type of measurement. The conclusion derives implications and recommendations for future research that aims to answer specific research questions regarding direct democracy or is interested in the measurement of direct democracy in general.
		\end{abstract}
	    \vspace*{2mm}
        \center		
		{\large {Seminar: Representative, direct and cooperative participation in comparison} \par}
		
	\end{titlepage}
}

%%% automated table of contents
\newcommand{\contents}{
	\newpage
	\thispagestyle{empty}
	\vspace{20mm}
	\tableofcontents
}



%%% Title page
\maketitle
\newpage
\contents
\clearpage
\listoffigures
\clearpage
\listoftables
\clearpage

%\clearpage
%
%%<!-- Inhaltsverzeichnisse -->
%\thispagestyle{empty}
%\setstretch{1.15}
%\tableofcontents
%\listoffigures
%
%\clearpage
%\setstretch{1.44}
%<!-- \onehalfspacing -->