\documentclass[]{article}
\usepackage{lmodern}
\usepackage{amssymb,amsmath}
\usepackage{ifxetex,ifluatex}
\usepackage{fixltx2e} % provides \textsubscript
\ifnum 0\ifxetex 1\fi\ifluatex 1\fi=0 % if pdftex
  \usepackage[T1]{fontenc}
  \usepackage[utf8]{inputenc}
\else % if luatex or xelatex
  \ifxetex
    \usepackage{mathspec}
  \else
    \usepackage{fontspec}
  \fi
  \defaultfontfeatures{Ligatures=TeX,Scale=MatchLowercase}
\fi
% use upquote if available, for straight quotes in verbatim environments
\IfFileExists{upquote.sty}{\usepackage{upquote}}{}
% use microtype if available
\IfFileExists{microtype.sty}{%
\usepackage{microtype}
\UseMicrotypeSet[protrusion]{basicmath} % disable protrusion for tt fonts
}{}
\usepackage[margin=1in]{geometry}
\usepackage{hyperref}
\hypersetup{unicode=true,
            pdfborder={0 0 0},
            breaklinks=true}
\urlstyle{same}  % don't use monospace font for urls
\usepackage{graphicx,grffile}
\makeatletter
\def\maxwidth{\ifdim\Gin@nat@width>\linewidth\linewidth\else\Gin@nat@width\fi}
\def\maxheight{\ifdim\Gin@nat@height>\textheight\textheight\else\Gin@nat@height\fi}
\makeatother
% Scale images if necessary, so that they will not overflow the page
% margins by default, and it is still possible to overwrite the defaults
% using explicit options in \includegraphics[width, height, ...]{}
\setkeys{Gin}{width=\maxwidth,height=\maxheight,keepaspectratio}
\IfFileExists{parskip.sty}{%
\usepackage{parskip}
}{% else
\setlength{\parindent}{0pt}
\setlength{\parskip}{6pt plus 2pt minus 1pt}
}
\setlength{\emergencystretch}{3em}  % prevent overfull lines
\providecommand{\tightlist}{%
  \setlength{\itemsep}{0pt}\setlength{\parskip}{0pt}}
\setcounter{secnumdepth}{0}
% Redefines (sub)paragraphs to behave more like sections
\ifx\paragraph\undefined\else
\let\oldparagraph\paragraph
\renewcommand{\paragraph}[1]{\oldparagraph{#1}\mbox{}}
\fi
\ifx\subparagraph\undefined\else
\let\oldsubparagraph\subparagraph
\renewcommand{\subparagraph}[1]{\oldsubparagraph{#1}\mbox{}}
\fi

%%% Use protect on footnotes to avoid problems with footnotes in titles
\let\rmarkdownfootnote\footnote%
\def\footnote{\protect\rmarkdownfootnote}

%%% Change title format to be more compact
\usepackage{titling}

% Create subtitle command for use in maketitle
\newcommand{\subtitle}[1]{
  \posttitle{
    \begin{center}\large#1\end{center}
    }
}

\setlength{\droptitle}{-2em}
  \title{}
  \pretitle{\vspace{\droptitle}}
  \posttitle{}
  \author{}
  \preauthor{}\postauthor{}
  \date{}
  \predate{}\postdate{}


\begin{document}

\begin{table}[!h]
    \centering
    \caption{Overview of the Selected Datasets}
    \label{datasets}
    \resizebox{\textwidth}{!}{%
        \begin{tabular}{@{}llllcc@{}}
            \toprule
            \textbf{Dataset} & \textbf{Level} & \textbf{Types} & \textbf{Available Information} & \textbf{Subnationality} & \textbf{Scope} \\ \midrule
            \textit{IDEA} & \begin{tabular}[c]{@{}l@{}}Country-\\ Level\end{tabular} & \begin{tabular}[c]{@{}l@{}}Mandatory and\\ Optional referendum,\\ Citizens’ Initiative,\\ Agenda Initiative, \\ Recall\end{tabular} & \begin{tabular}[c]{@{}l@{}}Primarily Rules in Form\\ \\ Top-down/bottom-up \\ not separated\\ \\ Information on different quora/ \\ thresholds and bindingness\\ \\ Data preparation necessary\end{tabular} & \begin{tabular}[c]{@{}l@{}}yes, but limited \\ information\end{tabular} & \begin{tabular}[c]{@{}l@{}}198 Countries,\\ up-to-date\end{tabular} \\ \midrule
            \textit{V-Dem} & \begin{tabular}[c]{@{}l@{}}Country-\\ Year\end{tabular} & \begin{tabular}[c]{@{}l@{}}Obligatory Referendum,\\ Plebiscite,\\ (Facultative) Referendum,\\ Citizens' Initiative\end{tabular} & \begin{tabular}[c]{@{}l@{}}Rules in Form and Use\\ \\ Top-down/bottom-up \\ can be differentiated\\ \\ Information on different quora/\\ thresholds, and bindingness\\ \\ Data can be used \\ for quantitative research \\ without much preparation\end{tabular} & \begin{tabular}[c]{@{}l@{}}yes, but limited \\ information\end{tabular} & \begin{tabular}[c]{@{}l@{}}178 Countries,\\ 1900 - 2016\end{tabular} \\ \midrule
            \textit{sudd} & \begin{tabular}[c]{@{}l@{}}Referendum-\\ Level\end{tabular} & \begin{tabular}[c]{@{}l@{}}Inofficial Vote,\\ Plebiscite,\\ Obligatory Referendum, \\ Facultative Referendum,\\ Initiative\end{tabular} & \begin{tabular}[c]{@{}l@{}}Primarily Rules in Use\\ \\ Top-down/bottom-up \\ can be differentiated\\ \\ Information on bindingness, \\ quora and thresholds depend \\ on each popular vote\\  \\ Data preparation necessary\end{tabular} & yes & \begin{tabular}[c]{@{}l@{}}140 Countries on \\ national level,\\ 1791 - present\end{tabular} \\ \midrule
            \textit{\begin{tabular}[c]{@{}l@{}}Direct\\ Democracy\\ Navigator\end{tabular}} & \begin{tabular}[c]{@{}l@{}}Country-\\ Level\end{tabular} & \begin{tabular}[c]{@{}l@{}}Agenda Initiative,\\ Initiative,\\ Referendum,\\ Plebiscite\\ \\ (with subtypes for \\ the latter three)\end{tabular} & \begin{tabular}[c]{@{}l@{}}Top-down/bottom-up \\ can be differentiated\\ \\ Information initiating \\ authority and on the author(s) \\ of the ballot proposal(s)\\ \\ Information on bindingness of \\ plebiscites, quora and\\ thresholds not available\\ \\ Data preparation necessary, \\ incomplete data\end{tabular} & yes & \begin{tabular}[c]{@{}l@{}}112 Countries, \\ with their regions \\ and munincipalities\\ \\ up-to-date, \\ but incomplete\end{tabular} \\ \bottomrule
        \end{tabular}%
    }
\end{table}

Whenever the effects of direct democracy are meant to be evaluated by
researchers, they first need to gather (systematic) data that expresses
the degree of direct democracy. Before the measurement approaches and
their indicator construction is discussed, we therefore describe data
sources on direct democratic institutions that allow for cross-national
comparison. Historically, data on direct democracy was mostly available
for two regions: Switzerland and the individual states of the United
States of America. It wasn't until Butler and Ranney's attempt at
collecting information on referendums from all over the world that
measuring direct democracy became comparable across a wide variety of
countries {[}cf. @butler1978referendums{]}. Since then, established
worldwide democracy indices (for example Polity IV or Freedom House)
have been fairly neglectant on the topic of direct democracy and do not
include such variables in their measurements. Other individual
researchers have been more focused on specific regions, for example
Latin America {[}cf. @lissidini2011democracia pp. 60-67;
@zovatto2015instituciones; @madronal2005{]}, Asia {[}cf. @hwang2005{]}
and Central and Eastern Europe {[}cf. @auer2001direct{]}.

The following section introduces three datasets that include
institutional provisions and/or usage of direct democracy for countries
around the world: \emph{Varieties of Democracy} (short: V-Dem), the
\emph{Direct Democracy Database} by the Institute for Democracy and and
Electoral Assistance (short: IDEA), and the \emph{Democracy Navigator}
operated by the Research Center of Citizen Participation/Institute for
Democracy And Participation Research of the University of Wuppertal. A
different database that would also have been worthy of including in our
comparison is compiled by the Research and Documentation Centre on
Direct Democracy (\emph{c2d}), however at the time of writing this
paper, the website containing the data has been offline and thus
inaccessible (as of February and March 2018). We therefore rely on the
\emph{Database and Search Engine for Direct Democracy} (short: sudd)
compiled by Beat Müller, which was translated from German into English
by c2d. The \emph{Democracy Barometer} is not discussed at this point,
as it only includes superficial data for provision and use of direct
institutions (compared to the other data sources), namely a total of
three already aggregated indicators.

Before delving deeper into the separate datasets, it would be wise to
identify certain common dimensions along which they can be classified.
For example, the examined datasets differ in in their scope (temporal/
geographical), in terms of which direct democratic mechanisms they
provide data for and whether it can be used to account for dimensions
like top-down/bottom-up, easiness or decisiveness. Also, the datasets
vary in whether they are composed of cross-sectional or time-series
country-level data or are based on individual referendums, which is
especially relevant for the distinction of rules in form and rules in
use. All of the discussed features imply that the data sources might be
more or less suitable for researchers, depending on the desired
operationalization of direct democracy. Table \ref{datasets} gives a
short overview of the selected datasets.

\subsection{Varieties of Democracy - Direct Democracy
variables}\label{varieties-of-democracy---direct-democracy-variables}

The most comprehensive and systematic approach might be included in the
Varieties of Democracy dataset (v7). It includes data in the range from
1900 to 2016 for a total of 178 countries {[}@coppedge2017v pp.
137-153{]}. The codebook specifically addresses the top-down and
bottom-up differentiation in their typology of direct democratic
institutions, which makes comparison between these dimensions
convenient. The covered top-down institutions are obligatory referendums
and plebiscites, initiatives and (facultative) referendums are featured
as bottom-up mechanisms. The V-Dem data is in a country-year format so
that it provides the legal provisions for each institution in a given
year. Concerning rules in use, V-Dem includes a variable for each
institution, capturing the number of occurrences in a specific year.
Moreover, a variable is included which indicates if there ways any
``credible'' use of a direct popular vote institution in a given year
(by credible it is referred to whether the official results reflect the
actual vote). The V-Dem data refers almost completely to the national
level and does not include any subnational forms of direct democracy,
besides one item that asks whether the respective institution exists at
national, subnational, or at both levels (except for obligatory
referendums). Besides information about the existence of institutions,
it is assessed whether they are binding and whether certain quora apply
(approval and turnout quorum, administration threshold and supermajority
requirements for every institution as well as signature requirements and
gathering periods for the bottom-up institutions).

\subsection{Direct Democracy Database
(IDEA)}\label{direct-democracy-database-idea}

The direct democracy database by \emph{IDEA} can be seen as collection
of information rather than a dataset that is meant to be used in
quantitative analysis. It does not have a specified temporal range,
instead it simply features the most recent available information. It
includes several columns assessing provisions of direct democracy in a
total of 198 countries, but mostly focuses on rules in form (it only
addresses rules in use by including information on the first referendum
or initiative held in a specific country and whether a national
referendum was held since 1980). The IDEA database covers legal
provisions for five institutions: \emph{mandatory/obligatory
referendums}, \emph{optional referendums}, \emph{citizens' initiatives},
\emph{agenda initiatives} and recall. This information is available for
198 countries also includes regional and local levels, though only
general information about the existence of such provisions is provided.
A greater range of information on the respective institutions is only
available on the national level. For referendums, the information
consists of restricted/allowed issues, possible initiators, who drafts
the referendum question, who decides the final form of the ballot text,
approval quorum, majority requirements and bindingness. A disadvantage
in the structure of the IDEA data is that its underlying typology of
institutions does not differentiate between citizen initiated optional
referendums and top-down optional referendums/plebiscites (though
extraction is possible and is performed in Chapter \ref{empiric}).
Moreover optional and mandatory referendums are mixed up in the same
column, resulting in strings like ``Mandatory referendum - always/
Optional referendum - sometimes''. This makes it challenging to use the
data in quantitative analysis, especially if one is interested in data
for a large number of countries. Regarding initiatives, the same
mixed-up data structure applies for agenda and citizen' initiatives. For
both initiatives information is available for issues voted on, required
materials, disqualification, legality checks, the author of the
initiatives title and ballot text. For recall institutions, informations
are provided in similar fashion. Regarding signature collection,
detailed information is available, for example on time gathering periods
and requirements, although different institutions are again captured in
the same variable (this time optional referendums, agenda initiatives,
citizen initiatives and recall), making it difficult to extract the
information for quantitative analysis. In general, this coding induces
uncertainty whether some entries that lack further specification refer
to all of the provided institutions or not.

\subsection{Database and Search Engine for Direct Democracy
(sudd)}\label{database-and-search-engine-for-direct-democracy-sudd}

Similarly to the direct democracy database by IDEA, the \emph{Database
and Search Engine for Direct Democracy} (sudd) is not inherently
designed to be used in quantitative analyses. However, it includes a
wide range of data on all kinds of actual direct popular votes from 1791
to the current day, making it quite valuable for researchers seeking to
explore direct democratic mechanisms over time. The unit of analysis for
the sudd data is any direct popular vote that occurred in any given year
in any given country or region (autonomous/independent/unrecognized or
colonized regions, does not refer to subnational level, eg. Swiss
Cantons). This leads to a total of 2815 referendums in 400 countries and
regions. Given the structure of the data, rules in form are not
measurable (or only per referendum), making it mostly useful for
research assessing rules in use. For each popular vote, different
variables are captured, for example question type, majority
requirements, number of people entitled to vote and the results of the
vote. Also majority requirements and sometimes participation or approval
quora are documented. Rather useful information is captured in the
column \emph{type of vote}, which categorizes amongst others the type of
vote (\emph{unofficial}, \emph{plebiscite}, \emph{obligatory
referendum}, \emph{facultative referendum} and \emph{initiative}), who
initiated it (for example parliament, president ot the people) and the
bindingness. Again, if one wishes to use the data for quantitative
analysis, the information has to be extracted with some effort. The
coding scheme makes it challenging to use without some extensive data
cleaning and transforming and, given that the data is not available for
download, it has to be web scraped from the website itself.\footnote{On
  request, Beat Müller kindly offered to provide the Data in XML format
  after restructuring, which would possibly have made data preparation
  somewhat easier, however such data was not accessible before the end
  of March 2018. Nevertheless there are some issues which will be
  addressed in the empirical section, when discussing the construction
  of rules in use indicators.}

\subsection{The Navigator of Direct
Democracy}\label{the-navigator-of-direct-democracy}

A noteworthy data compilation is the \emph{Direct Democracy Navigator}
database, although it is still under development. The aim of the
Democracy Navigator is to feature available legal designs, practices and
and events of direct democracy in all jurisdictions around the world. In
their classification typology, it also accounts for the (veto player or
minority) status of a top-down initiator as well as the author of the
ballot proposal(s) {[}cf. @demnavtypology{]}, which results in nine
different subtypes of direct democratic votes, with three overarching
main types:

\vspace{0.2cm}

\begin{tabular}{p{13.5cm}}
    1.) \textit{initiative} (citizens' initiative, citizens' initiative + authorities' counter-proposal, agenda setting initiativ)
    
    2.) \textit{referendum} (citizen-initiated referendum, citizen-initiated referendum + counter-proposal, obligatory referendum)
    
    3.) \textit{plebiscites} (plebiscite, veto-plebiscite, authorities' minority plebiscite, authorities' minority veto-plebiscite)
    
\end{tabular}

Unfortunately, the data available for legal provisions seems to be
rather incomplete and is not available as a dataset, even on the
national level (thus it needs to be web scraped). Moreover, it is not
clear at all whether not-listed legal provisions are actually not
provided or rather missing values. Furthermore, sometimes types of
direct democratic votes are counted twice without obvious reasons (for
example the agenda initiative in Liechtenstein) or because they are
separately regulated for different issues (for example plebiscites in
Taiwan). All of this makes it difficult to use for quantitative
comparative analyses, although the database is useful for cross-checking
information with other sources.

In general, there is no recommendation as to which of the data sources a
quantitative researcher should use in their empirical analysis. This
decision depends heavily on the research question to be answered and the
underlying concept of direct democracy. For example, many studies
exclude the recall mechanism from their measurement, because it can be
seen as a accountability function of representative democracy and not an
element of direct democracy. If one is interested in examining
provisions for institutions in regard to whether they are top-down or
bottom up initiated, a researcher is better advised in using the V-Dem
data, as it takes much more effort to obtain the information from other
sources such as IDEA. Generally speaking, of the discussed data sources
V-Dem is the most convenient and user-friendly dataset to use in
quantitative analyses, mostly numeric in nature and available in
time-series format since 1900, with the drawback of some unavailable
information (for example on agenda initiatives or the authors of ballot
proposals). On the other hand, the IDEA Database offers more detailed
documentation of legal provisions and the corresponding constitutional
paragraphs are often provided as background information, which makes it
a valuable data source as well. The Democracy Navigator is a rather
detailed database as well, which is especially useful for cross-checking
information or to gain additional information if a researcher is
interested in legal designs in regard to the status of the initiating
authority or the author of ballot proposals. Lastly, the sudd database
should be considered by researchers who are specifically interested in
the occurrence of referendums and other mechanisms in practice (on the
national level but also for independent and dependent regions).


\end{document}
