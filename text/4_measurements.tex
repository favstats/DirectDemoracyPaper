\documentclass[]{article}
\usepackage{lmodern}
\usepackage{amssymb,amsmath}
\usepackage{ifxetex,ifluatex}
\usepackage{fixltx2e} % provides \textsubscript
\ifnum 0\ifxetex 1\fi\ifluatex 1\fi=0 % if pdftex
  \usepackage[T1]{fontenc}
  \usepackage[utf8]{inputenc}
\else % if luatex or xelatex
  \ifxetex
    \usepackage{mathspec}
  \else
    \usepackage{fontspec}
  \fi
  \defaultfontfeatures{Ligatures=TeX,Scale=MatchLowercase}
\fi
% use upquote if available, for straight quotes in verbatim environments
\IfFileExists{upquote.sty}{\usepackage{upquote}}{}
% use microtype if available
\IfFileExists{microtype.sty}{%
\usepackage{microtype}
\UseMicrotypeSet[protrusion]{basicmath} % disable protrusion for tt fonts
}{}
\usepackage[margin=1in]{geometry}
\usepackage{hyperref}
\hypersetup{unicode=true,
            pdfborder={0 0 0},
            breaklinks=true}
\urlstyle{same}  % don't use monospace font for urls
\usepackage{graphicx,grffile}
\makeatletter
\def\maxwidth{\ifdim\Gin@nat@width>\linewidth\linewidth\else\Gin@nat@width\fi}
\def\maxheight{\ifdim\Gin@nat@height>\textheight\textheight\else\Gin@nat@height\fi}
\makeatother
% Scale images if necessary, so that they will not overflow the page
% margins by default, and it is still possible to overwrite the defaults
% using explicit options in \includegraphics[width, height, ...]{}
\setkeys{Gin}{width=\maxwidth,height=\maxheight,keepaspectratio}
\IfFileExists{parskip.sty}{%
\usepackage{parskip}
}{% else
\setlength{\parindent}{0pt}
\setlength{\parskip}{6pt plus 2pt minus 1pt}
}
\setlength{\emergencystretch}{3em}  % prevent overfull lines
\providecommand{\tightlist}{%
  \setlength{\itemsep}{0pt}\setlength{\parskip}{0pt}}
\setcounter{secnumdepth}{0}
% Redefines (sub)paragraphs to behave more like sections
\ifx\paragraph\undefined\else
\let\oldparagraph\paragraph
\renewcommand{\paragraph}[1]{\oldparagraph{#1}\mbox{}}
\fi
\ifx\subparagraph\undefined\else
\let\oldsubparagraph\subparagraph
\renewcommand{\subparagraph}[1]{\oldsubparagraph{#1}\mbox{}}
\fi

%%% Use protect on footnotes to avoid problems with footnotes in titles
\let\rmarkdownfootnote\footnote%
\def\footnote{\protect\rmarkdownfootnote}

%%% Change title format to be more compact
\usepackage{titling}

% Create subtitle command for use in maketitle
\newcommand{\subtitle}[1]{
  \posttitle{
    \begin{center}\large#1\end{center}
    }
}

\setlength{\droptitle}{-2em}
  \title{}
  \pretitle{\vspace{\droptitle}}
  \posttitle{}
  \author{}
  \preauthor{}\postauthor{}
  \date{}
  \predate{}\postdate{}


\begin{document}

After examining available data on direct democracy, we discuss how such
data can be aggregated into general assessments of direct democracy. We
therefore introduce measurement approaches derived from contemporary
research and examine them in regard to which dimensions they cover in
which way. First, we elaborate on measurement approaches assessing
direct democratic rules in form and rules in use separately, starting
with the former. Then, two indices that cover both rules in use and
rules in form in an aggregated way are introduced. Table
\ref{measure_overview} gives an overview of all approaches in regard to
which mechanisms they include and whether they account for easiness and
decisiveness in their aggregation (with a weighting procedure), and
gives some general notes on the indices' calculation.

\subsection{Approaches assessing Rules in
Form}\label{approaches-assessing-rules-in-form}

This section introduces the three rules in form measures, which are
compared in this paper: @gherghina2016, @peters2016, and the Legal
Provisions for Direct Democracy subcomponent of the Democracy Barometer
{[}@dembardata{]}. A simple count measure of Legal Designs, for which
data was derived from the Direct Democracy Navigator (which is also used
as a comparison indicator), is not included here, as it is not
considered a \emph{measurement approach} in the actual sense and not
derived from literature, but instead constructed in an ad hoc manner.

\subsubsection{Gherghina 2016}\label{gherghina-2016}

A current measurement approach explicitly dealing with rules in form
comes from @gherghina2016 pp.~7-9, who relies on data primarily from the
IDEA Database. Five direct democratic institutions are covered:
Obligatory and optional referendums, citizens and agenda initiative as
well as recall. The index relies solely on their existence, ranging from
0 - 5, with each point standing for the existence of one of the
institutions. A second index is constructed which measures the level of
subnational direct democracy in the same manner, adding up the scores
for regional and local direct democracy (ranging from 0-10). However,
there is no differentiation between top-down and bottom-up institutions
and a disadvantage of the approach taken by Gherghina is that neither
decisiveness nor easiness are part of the measurement. On the upside,
this also means that the indicator is rather effortlessly constructed
and allows for a straightforward interpretation. On the other hand, one
might call into question if the values of such a count index are
equidistant and if each mechanism should count the same, for example,
does the existence of a top-down mechanism imply the exact same
importance than a bottom-up mechanism? Unfortunately, Gherghina does not
aggregate both indices into one general index of direct democratic rules
in form.

\subsubsection{Peters 2016}\label{peters-2016}

Another approach in assessing rules in form comes from @peters2016, who
examines the effects of direct democracy on party membership. She
differentiates between bottom-up and top-down direct democracy,
calculating an aggregated measure based on two sub-indices, all ranging
between 0 and 1.5. Unfortunately, Peters' description of her
construction of the bottom-up indicator is quite ambiguous and thus
reproducing her measure for the empirical comparison became a difficult
task. For this reason, there will be a thorough discussion of her
measurement approach.

As described in the variable description in the study's appendix
{[}@peters2016 pp. 155{]}, the top-down direct democracy indicator is
composed of the variables constitutional referendum (0 = no institution;
2 = institution exists) and legislative referendum (0 = no institution;
0.5 = institution exists, but referendum is not binding; 1 = institution
exists and is binding). The result is described as five-point indicator
ranging ``from 0 to 1.5, where 0 implies that there are no top-down
structures and 1.5 means that there are provisions for both
constitutional and binding legislative referendums'' {[}@peters2016 pp.
144{]}. As it turns out, the indicator can take six possible values,
though the difference could be caused by the fact that in Peters sample
of 16 countries, there is no country that solely provides the
possibility of a constitutional referendum, and therefore there is no
empirical value of 2 or respectively 1 in the rescaled indicator (for
the data description see the online appendix of @peters2016, which can
be found on the authors website). The other empirical values (as shown
in Peters' Figure 1, pp.~147), correspond to an aggregation method of
adding the first two variables and dividing the sum by 2.

According to the variable description, the bottom-up indicator of direct
democracy is a ``three-point index composed of `citizen initiative' (0 =
no institution; 3 = institution exists (1.5 = when it exists but is
limited, e.g., to existing laws only)) and `agenda initiative' (0 = no
institution; 1.5 = institution exists)'' {[}@peters2016 pp. 155{]}.
Again, there is a difference in possible combinations of the variables
and the description of a ``three-point index'', which could also be
explained by the fact that there are only three empirical values. On the
other hand, when reading the descriptions, appendix and online appendix
it's not easily comprehended how exactly the indicator was created. When
interpreting the empirical results for the bottom-up indicator, Peters
describes the values in parentheses {[}@peters2016 pp. 147{]}: ``popular
initiatives (a score of 1.5)'' and ``agenda initiatives (a score of
0.75)''. The interpretation does not correspond to the aforementioned
description of the bottom-up index as ``ranging between 0 and 1.5, where
0 implies that there are no provisions for either an agenda or citizen
initiative and \emph{1.5 means that there are provisions for both}''
{[}@peters2016 pp.~145; emphasis by the authors{]}.

\begin{landscape}
    \begin{table}[]
        \centering
        \scriptsize
        \caption{Overview of Used Direct Democracy Measures}
        \label{measure_overview}
        \begin{tabular}{@{}llllll@{}}
            \toprule
            \textbf{Author/Main Source} & \textbf{Mechanisms} & \textbf{Easiness} & \textbf{Decisiveness} & \textbf{Calculation} & \textbf{Values} \\ \midrule
            \multicolumn{6}{c}{\textit{Rules in Form}} \\ \midrule
            \multicolumn{1}{l}{Gherghina 2016} & \begin{tabular}[c]{@{}l@{}}Mandatory and\\ Optional Referendum,\\ Citizens’ Initiative, \\ Agenda Initiative, \\ Recall\end{tabular} & \multicolumn{1}{c}{-} & \multicolumn{1}{c}{-} & Counts existence of legal provisions for each mechanism. & \begin{tabular}[c]{@{}l@{}}0 - 5,\\ discrete\end{tabular} \\ \midrule
            \multicolumn{1}{l}{Democracy Barometer} & \begin{tabular}[c]{@{}l@{}}Mandatory Referendum, \\ Veto-Player Referendum,\\ Popular Veto, \\ Popular Initiative\end{tabular} & \begin{tabular}[c]{@{}l@{}}only ease of \\ approval\end{tabular} & \begin{tabular}[c]{@{}l@{}}only\\ binding \\ mechanisms\end{tabular} & \begin{tabular}[c]{@{}l@{}}Counts existence of legal provisions for each mechanism, \\ multiplied with a quorum variable.\end{tabular} & \begin{tabular}[c]{@{}l@{}}0 - 4,\\ continuous\end{tabular} \\ \midrule
            \multicolumn{1}{l}{Peters 2016} & \begin{tabular}[c]{@{}l@{}}Referendums,\\ Plebiscites, \\ Citizens' Initiatives,\\ Agenda Initiatives\end{tabular} & \multicolumn{1}{c}{-} & \begin{tabular}[c]{@{}l@{}}non-binding \\ plebiscites \\ count half\end{tabular} & \begin{tabular}[c]{@{}l@{}}Direct Democracy Index: \\ \textit{Bottom-Up + Top-Down}\end{tabular} & \begin{tabular}[c]{@{}l@{}}0 - 1.5,\\ discrete\end{tabular} \\ \midrule
            \multicolumn{6}{c}{\textit{Rules in Use}} \\ \midrule
            \multicolumn{1}{l}{Gherghina 2016} & \begin{tabular}[c]{@{}l@{}}referendums\\ ("issues put \\ to a vote")\end{tabular} & \begin{tabular}[c]{@{}l@{}}only ease of \\ approval\end{tabular} & \begin{tabular}[c]{@{}l@{}}non-binding \\ referendums \\ count half\end{tabular} & \begin{tabular}[c]{@{}l@{}}Summed up over 19-year period.\\ \\ Non-binding mechanisms receive a 0.5 weight \\ and score is multiplied with quorum variable.\end{tabular} & continuous \\ \midrule
            \multicolumn{1}{l}{Blume et al. 2007} & \begin{tabular}[c]{@{}l@{}}Mandatory Referendum,\\ Optional (Citizen Initiated) \\ Referendum,\\ Initiative\end{tabular} & \multicolumn{1}{c}{-} & \multicolumn{1}{c}{-} & \begin{tabular}[c]{@{}l@{}}Summed up over 10-year period.\\ \\ Categorization into four categorie.\end{tabular} & \begin{tabular}[c]{@{}l@{}}0-3,\\ discrete\end{tabular} \\ \midrule
            \multicolumn{1}{l}{\begin{tabular}[c]{@{}l@{}}Either sum or average: \\ Bauer \& Fatke 2014,\\ Peters 2016,\\ Democracy Barometer\end{tabular}} & \begin{tabular}[c]{@{}l@{}}Depends on \\ Measurement\end{tabular} & \multicolumn{1}{c}{-} & \multicolumn{1}{c}{-} & \begin{tabular}[c]{@{}l@{}}Sum or average of the occurrence of mechanisms \\ (within a determined time-frame), followed by logarithmization.\end{tabular} & continuous \\ \midrule
            \multicolumn{6}{c}{\textit{Mixed Approaches}} \\ \midrule
            \multicolumn{1}{l}{\begin{tabular}[c]{@{}l@{}}Fiorino 2017,\\ DDI\end{tabular}} & \begin{tabular}[c]{@{}l@{}}Citizens' Initiative, \\ Agenda Initiative, \\ Facultative Referendum,\\ Obligatory Referendum,\\ Plebiscite\end{tabular} & \begin{tabular}[c]{@{}l@{}}both ease \\ of initiation \\ and approval\end{tabular} & \begin{tabular}[c]{@{}l@{}}accounted \\ for\end{tabular} & Qualitative rating into seven categories (exact aggregation method unclear). & \begin{tabular}[c]{@{}l@{}}1-7,\\ discrete\end{tabular} \\ \midrule
            \multicolumn{1}{l}{\begin{tabular}[c]{@{}l@{}}Altman 2017,\\ DPVI (V-Dem)\end{tabular}} & \begin{tabular}[c]{@{}l@{}}Obligatory\\ Referendum,\\ Plebiscite,\\ (Facultative) Referendum,\\ Citizens' Initiative\end{tabular} & \begin{tabular}[c]{@{}l@{}}both ease\\ of initiation \\ and approval\end{tabular} & \begin{tabular}[c]{@{}l@{}}accounted \\ for\end{tabular} & \begin{tabular}[c]{@{}l@{}}For each mechanism: \\ \textit{Ease of Initiation + Ease of Approval} $\times$ \textit{Consequences}\\ \\ For detailed explanation see Altman 2017.\end{tabular} & \begin{tabular}[c]{@{}l@{}}0-1,\\ continuous\end{tabular} \\ \bottomrule
        \end{tabular}
    \end{table}
\end{landscape}

If the variable agenda initiative can take the values 0 and 1.5, and
citizen initiative a maximum of 3, it could again be the case that the
sample only had countries with either agenda or citizen initiatives, but
not both. Table A in Peters online appendix lists two countries with
agenda initiatives (Austria and Netherlands) and two countries (Italy
and Switzerland) with the description ``agenda/popular initiative'',
which implies that both institutions exist there.\footnote{Cross-checking
  with the IDEA Database shows that Italy has both agenda and citizens
  initiative, while for Switzerland only the citizens initiative, but
  not the agenda initiative ist present. The agenda initiative
  description of IDEA states that Switzerland provides ``little more
  than a right to petition any government organ or legislature to do
  something - pass legislation, change the constitution, etc. The
  respective agency or organ is not under a legal obligation to consider
  such proposals''.\footnote{see:
    \url{https://www.idea.int/node/205203})} Peters conception of
  bottom-up direct democracy includes ''that citizens themselves can
  organise the collection of signatures in order to {[}\ldots{}{]} force
  the legislature or government to address an issue they put forward''
  {[}@peters2016 pp. 144{]}, which might be less restrictive than the
  IDEA classification.} Unfortunately, there is no country in Peters'
sample that only offers the possibility of a citizen/popular initiative,
but not the agenda initiative, which would enable a more precise
reconstruction of the aggregation method.

Considering the variable descriptions and the appendices, the
\emph{first possibility} is that Peters did not aggregate the bottom-up
indicator out of the variables citizens initiative and agenda initiative
through addition (as the top-down indicator appears to be aggregated),
but instead in a way that ignores agenda initiatives in the case that
citizens initiatives are provided. Given sample data with only both
institutions, the variable takes values from 0 for no institution, 1.5
for only agenda initiatives and 3 for citizens initiatives (or
respective values of 0.75 and 1.5 on a variable ranging from 0 to 1.5),
corresponding to the parenthesized values in Peters interpretation text.
We are not sure whether this is the case, as the construction of the two
variables for the bottom-up indicator is composed of different values
for the existence of each institution, which implies a deliberate
weighting. For example, agenda initiatives are given only half the
weight of the initial values of citizens initiatives, much like the
mandatory referendum which counts twice as much as plebiscites.
Additionally, it would be completely unclear how the score for a
hypothetical country with a citizens' initiative value of 1.5 and an
agenda initiative value of 1.5 would be aggregated, being the only value
in the index actually reflecting ``both institutions''. A construction
ignoring agenda initiatives once citizens initiatives are provided
implies that once there is the latter, the former does not matter.
Legislation initated by referendum or by parliament are usually
considered as different institutions (e.g.~IDEA and V-Dem). Moreover,
the initial introduction of the bottom-up index describes the value 1.5
as the provision of both institutions, and the top-down indicator
appears to be constructed additively. A \emph{second possibility} is
that Peters assigned Italy and Switzerland the value 1.5 on the citizen
initiative variable, although in a cross-check with IDEA the institution
did not appear to be strongly ``limited'' in both countries, neither
does Peters' online appendix hint at that. \emph{Thirdly}, Peters could
have made a mistake when aggregating the bottom-up indicator, assigning
to countries with both provisions a value of 3 instead of the actual 4.5
(which would be the result of an addition). If so, the rescaling to a
range of 1.5 resulted in higher values for the agenda initiative, which
would result in a score of 0.5 (instead of the 0.75 specified by
Peters).

After constructing the bottom-up and top-down indicators ``the general
measure of direct democratic institutions is simply a combination of the
latter two measures; it is an eight-point index ranging from 0 (no
institutions) to 1.5 (constitutional, legislative and popular referendum
institutions)'' {[}@peters2016 pp. 145{]}. To sum up, the covered
institutions are constitutional referendums, plebiscites, citizen
initiatives and agenda initiatives. The decisiveness is captured by
applying only half the weight if legislative referendums are
non-binding. As both subindices range from 0 to 1.5, it appears that
there is no weighting between top-down and bottom-up institutions.
Within the subindices, obligatory referendums are assigned twice the
value of a binding legislative referendum, the same applies for citizen
compared to agenda initiatives. Neither ease of initiation nor ease of
approval are covered in the indicator, nor is the local/regional
dimension included. Whether citizens' initiatives are restricted to
certain issues is accounted for, as a ``limited'' provision of citizens
initiative gives only half of the maximum value. In general the index
has the advantage of a rather simple construction (besides the already
discussed issues with replicability), and the provision of separate
bottom-up and top-down indices.

\subsubsection{Democracy Barometer}\label{democracy-barometer}

In their index for the quality of democracy, the Democracy Barometer
provides a subcomponent called \emph{constitutional provisions for
direct democracy}, available in country-year format between 1990 and
2014, covering 68 countries {[}@dembarcodebook pp. 49{]}. The
subcomponent is composed of an indicator measuring direct democracy
provisions and as well as their respective required quora. The variables
are then aggregated by taking the arithmetic mean for the standardized
indicators. The first indicator consists of four direct democracy
mechanisms, each present institution improves the score by
one{[}@dembarcodebook pp. 49{]}:

\vspace{0.02cm}

\begin{tabular}{p{12.5cm}}
\footnotesize
    1.) \textit{Mandatory referendum} 
    
    2.) \textit{veto-player referendum:} referendum is triggered and question is asked by an existing veto-player
    
    3.) \textit{popular veto:} non veto-player triggers referendum, but question is asked by an existing veto player
    
    4.) \textit{popular initiative:} non veto-player asks question and triggers referendum
\end{tabular}

The underlying typology of institutions is rather different to the other
discussed approaches, as is it also considered whether a top-down
initiator is a political veto-player. The decisiveness is captured in so
far as only binding referendums are considered. Easiness of approval is
included in the second indicator, which measures constitutional
provisions for approval or participation quorum in direct democratic
votes (in case of different instruments, the quorum applying for most of
them is used). The variable is calculated by subtracting the quorum from
1, which means smaller values indicate less ease of approval. In case of
approval quorums, the value is multiplied with 2 before the subtraction,
because in order to meet an approval quorum of 25\%, at least 50\% of
the population must participate. Countries with no direct democracy are
given the value of the country with the highest quorum.

\subsection{Approaches assessing Rules in
Use}\label{approaches-assessing-rules-in-use}

The majority of approaches measure the rules in use by aggregating the
number of referendums and initiatives in the past by different
procedures, wich will now be discussed. This section introduces the
three measures considered for rules in use, which are derived from
@gherghina2016, summed measures {[}cf. @bauerfatke2014{]} and
categorized measures {[}cf. @blume2007{]}.

\textbf{Gherghina 2016}

Gherghina operationalizes the use of referendums at national level by
calculating the number of referendums from 1990-2008 (defined as a
question/issue put to a vote), but also including a weighting accounting
for decisiveness and easiness {[}cf. @gherghina2016 pp. 9{]}. Merely
consultative referendums are divided by two. The quorums are assessed
for binding and consultative referendums separately, with a ordinal
variable of three values: 1 = both turnout and approval quorums, 2 =
turnout quorum only, and 3 = no quorum required. This kind of
construction is an exception, as most studies assessing rules in use and
rules in form separately include easiness and decisiveness within their
measurement of formal rules or not at all. It is noteworthy that
Gherghina does not assess the ease of use, but only the ease of
approval, as he does not differentiate between top-down
referenda/plebiscites and bottom-up mechanisms when aggregating the
rules in use indicator.

\textbf{Summed Measures (e. g. Bauer/Fatke 2014)}

Besides Gherghina, the examined approaches assessing rules in use
separately all use either the sum or average of the occurrence of direct
democratic mechanisms within a determined time-frame. Bauer and Fatke
asses rules in use for the Swiss cantons by averaging the number of all
cantonal initiatives and optional referendums per year in the period
2002 to 2006 (mandatory referendums are excluded, as they don't fit the
authors theoretical argument) {[}cf. @bauerfatke2014 pp. 54{]}. As
Peters study relies on time-series data with four time-points in four
decades, she counted the number of referendums held for each decade
preceding the measurement of her dependent variable {[}cf. @peters2016
pp. 155{]}. A rather similar procedure is used in the Democracy
Barometer when assessing the effective use of direct democratic
instruments, which sums up the number of national non-mandatory
referendums per year {[}cf. @dembarcodebook pp. 47{]}. As a last step,
Peters indicator as well as the one calculated by the Democracy
Barometer are logarithmized. For count measures it might be
recommendable to use logged variables, especially since additional
referendums might be less relevant in countries where there already are
many referenda {[}cf.~dembarcodebook pp.~47{]}.

\textbf{Categorized Measures (e. g. Blume 2007)}

The approach of Blume et al. differs from the already discussed in that
it categorizes the number of direct democratic votes in the timeframe of
1996 to 2005) into four categories, ranging from ``0 = no factually
observed direct democracy'' to ``1 = low level of factually observed
direct democracy (i.e.~one or two votes)'' to ``2 = medium level of
factually observed direct democracy (i.e.~three to five votes)'' to ``3
= high level of direct democracy (i.e.~more than five votes)''
{[}@blume2007 pp. 14{]}.

\clearpage

\subsection{Approaches assessing Rules in Form and Rules in
Use}\label{approaches-assessing-rules-in-form-and-rules-in-use}

This section introduces measures that explicitly include both rules in
form and use. Two of such indices will be utilized: the measure created
by @fiorino2017 and the \emph{Direct Popular Vote Index} in the V-Dem
dataset.

\textbf{Direct Democracy Index}

An index that is supposed to measure direct democracy is the Direct
Democracy Index (DDI) by @fiorino2017, who use data from @kaufmann2004,
@madronal2005 and @hwang2005. Fiorino et al. apply Kaufmanns'
seven-point rating of direct democracy by consulting country reports,
covering a dataset of 87 countries for the period of 2000-2005 {[}cf.
@fiorino2017 pp. 14; see appendix XX for the original description by
@kaufmann2004{]}.\footnote{See @kaufmannwaters2004 for a different
  version of the index applied to European countries.} The seven
categories are:

\vspace{0.05cm}

\begin{tabular}{p{13.5cm}}
\small
    1.) \textit{radical democrat}
    
    2.) \textit{progressive}
    
    3.) \textit{cautious}
    
    4.) \textit{hesitant}
    
    5.) \textit{fearful}
    
    6.) \textit{beginner}
    
    7.) \textit{authoritarian}
\end{tabular}

Each country's score is based on a qualitative assessment of the direct
democratic quality of the political system. The mechanisms explicitly
referred to in the coding scheme, as described by @kaufmann2004
pp.~26-29, are \emph{citizens' and agenda initiatives}, \emph{obligatory
referendums}, \emph{facultative referendums} and \emph{plebiscites.} The
actual use of direct democratic institutions is explicitly included. The
index covers ``the procedures a political system provides in order to
propose, approve, amend, and delete laws through popular initiative and
referendums, as well the actual practices of direct democracy and the
general political condition a country experiences'' {[}@fiorino2017 pp.
148{]}. The easiness of initiation and approval is covered by the
consideration of entry hurdles, time limits and majority
requirements/quora. As Blume et al. point out, an important disadvantage
of the DDI is that ``the criteria used for weighing the different
criteria remain completely opaque and that it does not tell anything on
the relevance of institutional details'' {[}@blume2007 pp. 12{]}. The
actual weight of the dimensions bottom-up/top-down, subnational direct
democracy, easiness, decisiveness and de jure/de facto measures for the
index construction is therefore unclear, though some hints can be drawn
when consulting the categories descriptions. For example, the categories
7 and 6 are only ascribed to countries that embody bottom-up mechanisms
as well as obligatory referendums. Category 5 consists of countries that
have practical experience, but where the procedures have only
plebiscitary character {[}cf. @kaufmann2004 pp. 39{]}, so bindingness as
well as bottom-up mechanisms seem to be given a greater weight.

\textbf{Direct Popular Vote Index (DPVI)}

The most complex of the introduced measurements is constructed by
Altman, using V-Dem data {[}@altman2017; @coppedge2017v pp. 62f.,
138-153{]}. The concept to be measured consists not only of de jure
measures but also includes the actual use of direct democracy. In his
paper, Altman names his index \emph{Direct Democratic Practice
Potential} (DDPP), however the corresponding V-Dem index is referred to
\emph{Direct Popular Vote Index} (DPVI), which will be used in this
paper for empirical comparison. Mostly, the described variables and
aggregation method of the DDPP resembles the one described in the V-Dem
codebook for the DPVI, with one major exception that will be discussed
later on. The mechanisms covered are \emph{popular initiatives},
\emph{popular referendums}, \emph{obligatory referendums} and
\emph{plebiscites}, whereas the first two are citizen initiated (in our
terms bottom-up) and the latter two are considered top-down mechanisms.

Altman/V-Dem provide not only an aggregated index, but also sub-indices
for the bottom-up/top-down dimensions as well as a measure for each
mechanism itself. The aggregated score for each mechanism results from
the addition of the values for the variables ease of initiation and ease
of approval, which is then multiplied with a term called
\emph{consequentiality.} The ease of initiation consists of the
existence of the mechanism and, in case that it is initiated from the
people, the number of signature and time limits for gathering them. The
major difference between the DDPP and DPVI is that the latter also
considers whether the respective institutions exist on the national,
subnational level or both {[}@coppedge2017v pp. 62{]}.\footnote{There is
  no hint to the reason why the term is only included in the DPVI and
  not the DDPP. The subindices for the respective institutions, as
  described in the V-Dem codebook, do not include the variable in their
  description, which makes it even more puzzling.} Ease of approval is
measured by calculating the surface of the polygon determined by the
quora/thresholds for participation, approval and supermajority, and
multiplying the result with district majority. Altman chooses this
intricate method of calculating a polygon area for the construction of
his index because many less complex aggregation methods would not
adequately account for the interaction between different quora (for
example, an approval quorum of 20\% already implies a
participation/turnout quorum of 40\%) {[}cf. @altman2017 pp.
1213-1215{]}. The consequentiality variable consists of the
decisiveness, whether the vote is binding (a value of 1) or consultative
(0.75), as well as another variable called credible threat. Credible
threat is calculated based on the frequency and success of direct
popular votes in the past. The aggregated index results from the
addition of the weighted scores for each mechanism. The bottom up
mechanisms receive a weighting of 1.5 to account for the penalty the
index construction applies for signature thresholds and time gathering
periods (which are not relevant for top-down initiated mechanisms).
Lastly, the DPVI is standardized to range between 0 and 1. As the
construction of the DPVI is rather complex, we refrain from describing
it thoroughly and recommend the reader to consult @altman2017 for
further explanation.

A great advantage of the DPVI is the scope of the considered dimensions,
as well as the yearly coverage of 178 countries since 1900 and the
availability of these data. The decisiveness is accounted for, and also
the easiness, with an explicit distinction between ease of initiation
and approval and an elaborate aggregation method for different quora.
Moreover, separate indices for bottom-up and top-down mechanisms are
provided. Rules in use are not separately measured but implicitly
included in the variable measuring credible threat, although the purpose
of this variable is not to directly measure de facto direct democracy,
but rather the usage potential of a mechanism. With the advantage of the
broad scope of dimensions and complexity of the index aggregation comes
the drawback of a less straightforward interpretation of the aggregated
index and sub-indices.

In conclusion, all of the examined approaches have in common that their
measurements rely to some degree on the existence of a defined set of
institutions, and in case of rules in use assessments, whether these
institutions are used in practice within a certain time-frame. Some of
the approaches give these mechanisms equal weighting in aggregation, but
sometimes bottom-up institutions get a greater weight than top-down
mechanisms, binding referendums get assigned higher values than
consultative ones, or obligatory referendums weigh more than
plebiscites. Mostly, those values seem to be chosen rather arbitrarily,
as there is, for example no reference to draw from when to decide if
consultative mechanisms (in comparison with binding ones) should be
given the same value {[}cf. @gherghina2016{]}, a 0.75 weighting {[}cf.
@altman2017{]}, half the weight {[}cf. @peters2016{]}, or be not
considered at all {[}cf. @dembardata{]}. Moreover, the approaches differ
in regard to whether they consider the easiness of approval and/or
initiation and if so, in which way they are included in the index
aggregation. For example, the DPVI is constructed by assessing ease of
approval with a complex calculation of the surface of a polygon, which
is determined by three quorum variables, while @gherghina2016 classifies
quora with a categorical three-point variable. It is far from clear, if
and how these dimensions should be weighted to assess the degree of
direct democracy in a country most adequately, and besides @altman2017,
most authors do not spend much words on elaborating on their index
construction. In general, there is no recommendation in regard to which
index construction to use in one's empirical research, as it highly
depends on the research question.

Lastly, it has to be noted, that most of the discussed approaches are
only applied to democracies {[}e.g. @peters2016; @gherghina2016{]},
which makes the discussion of the \emph{democracy} part of direct
democracy unnecessary. Since we reconstruct the indicators for a range
of all available countries, this question has to be addressed. The
direct democratic mechanisms mentioned in this paper, even though called
\emph{democratic} are not necessarily limited to democratic countries.
Interestingly, some autocratic or semi-democratic regimes have adopted
direct popular vote instruments, using them to seemingly legitimate
their rule, a group of countries Altman labels the ``nightmare team''
{[}@altman2011 pp. 92{]}. Of the discussed measurement approaches, which
cover a wider range of countries, the rather qualitative rating of
@fiorino2017 accounts for the general state of democracy a country
experiences. The index provided by V-Dem itself is a sub-index of a much
broader assessment of democratic quality, and not accidentally called
Direct Popular Vote Index. Therefore, in the empirical comparison, we
examine available as well as reconstructed indices in relation to their
three-fold freedom house classification. An important notion for further
research is to find valid ways to aggregate indices assessing direct
popular votes and indicators representing the degree of democracy of a
given country. Some might argue that any direct democracy index must
account for the degree of authoritarianism or democracy in a country, or
else it does not capture the concept well. Others could argue that
authoritarian and semi-democratic regimes use direct popular votes for
other reasons than democracies but that doesn't mean that they should be
generally excluded from the analysis. It's not necessarily intuitive to
determine the weighting of the two dimensions in regard to each other,
and to decide whether the aggregation should be constructed
multiplicatively, additively or by another mathematical operation.


\end{document}
